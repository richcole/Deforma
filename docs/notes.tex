\documentclass[12pt]{article}

\usepackage{amsmath}
\usepackage{fullpage}

\title{OpenGL Game Programming Notes}
\author{Richard Cole}
\date{}

\begin{document}

\maketitle

\section{Journal}

\subsection{Sat Jul 20 11:01:34 PDT 2013}

I want to learn more about procedural generation of landscape and
voxel models. If we have say 100x100x100 region of voxels that works
out to be 1M voxels. 1000x1000x1000 works out to be 1G of voxels thus
voxels need to be merged into regions.

Perhaps the an easier place to start is with a height map. The first
question is how to render a height map. You can make 4 triangles
around a height map.

Start with a square (0, 0, 256, 256). Randomly assign heights to each
vertex. Add mid points at 4 midpoints at (128+-128, 128+-128). Assign
heights that are a pertabation from the mid point by a random
offset. Continue until width of the square is 1. One needs to ensure
that points are not written to more than once otherwise cliffs arise.

Another idea is to generate random gradients at different frequencies
and to then apply these to the terrain. This seems to be the idea of
perlin noise generators.





\subsection{Sat Jul 20 10:34:06 PDT 2013}

Movement is not smooth basically because movement occurs in the
simulation system and the simulation system is discrete. To smooth out
the movement things can be done with animation type effects. For
movement the user can have a direction and velocity, when rendered we
can take into account the time of the rendering.

This idea is extended to animation transtitions. When transitioning
from one animtation to the next one needs to interpolate between the
last active frame of the previous animation and the current animation.

Thus for a model we really want to know the current animation frame
and the next animation frame to interpolate between the two. The game
is based on a discrete simulation structure. Animations from models
need to be mapped onto ticks. A creature can make one decision per
tick. The players avitar similarly performs at most one action per
tick. The current simulation rate is 10 ms which seems a reasonable
rate at which to modify the avitars behaviour.

Currently loading the game artifacts takes quite a while. This is
actually time spend performing the gson deserialization. Combining the
texture together greatly increased the number of models which can be
rendered at a specific time. I need to move to a model where the scene
graph is loaded in the GPU memory and only small modifications are
made such as updating rotations of model skeletons.

There needs to be more preprocessing of the models to get them into a
format that is suitable for rendering. All textures need to be loaded
ahead of time.

Should add tool selection using the tab key. This will allow more
tools to be added to the game, so for example one can have a tool that
modifies the viewport parameters.

Should add gravity to keep the character grounded. Should also modify
the ground pattern to be a single cube rather than very many small
cubes.

Models are trees. The nodes have lists of translations and rotations
associated with them. A value alpha selects between two elements in
the lists and then interpolation is performed to find the actual
rotations and translation required.

We'd like this to occur on the GPU. For rendering using shaders there
are about 15 arrays that we can use. So for each point we'd like to
pass a single alpha value and then interp the position.

It might be easier to develop this code first in isolation and then
define a translation into the required format.

The question is then how to pass in the rotation and translation
matricies.

\section{Basic Mathematics}


This article is trying to get in place the basic mathematics to
understand how to use OpenGL.

We use row vectors, e.g. $x$, and denote associated column vector
$x\top$. We work in a homoginous coorindate system rather than the
cartesian space.

\[
   x = (x_1,\dots,x_4)
\]

The projection from homoginous coordinates to cartesian space is

\[
   x \mapsto \frac{1}{x_4}(x_1, x_2, x_3)
\]

Direction vectors have unit length. We can direction vectors for the 4
axis. $i_1,\dots,i_3$ for the three major direction. Matricies
are indicated by upper case.

\[
   M = \left ( \begin{array}{ccc}
     m_{1,1} & \dots & m_{4,1}  \\
     \vdots & \dots & \vdots  \\
     m_{4,1} & \dots & m_{4,4}
     \end{array}
   \right )
\]

The camera has a view point defined by a position, $p$,
forward direction $f$ and upward direction $u$. Moving
forward a scalar distance $d$ is

\[
   p = p + df
\]

The sideways direction $s$ is defined as $f \times u$. A translation
matrix for translation in the $x$ direction is

\[
  T(x) = \left ( 
  \begin{array}{cccc}
     1 & 0 & 0 & x_1      \\
     0 & 1 & 0 & x_2      \\
     0 & 0 & 1 & x_3      \\
     0 & 0 & 0 & 1
  \end{array}
  \right )
\]

Matricies are applied on the left so

\[
  T(x) p^\top = (p_1 + p_4 x_1, p_2 + p_4 x_2, p_3 + p_4 x_3, p_4)
\]

The projection back into 3 space gives

\[
  (p_i / p_4 + p_4 x_i / p_4)_{i} = (p_i/p_4 + x_i)_{i}
\]

The mouse moves in two dimensions $r = (r_1, r_2, 0, 0)$. We translate
these directions to angles $\theta = (\theta_1, \theta_2, \theta_3, 0)$. The
character has a position $p$.

Let $M(\theta, i)$ be a rotation around the axis $i$. The view
transform is then

\[
  T(p) R(\theta_2, s) R(\theta_1, u_1) 
\]

where $s = u_2 R(\theta_1, u_1)$.

This is the transform of the eye from it's initial position. For
OpenGL we need the inverse. Transformations and rotations have an easy
inverse you just negate vector or the angle.

\[
   (AB)^{-1} = B^{-1} A^{-1}
\]

Therefore we obtain the 

\[
  R(-\theta_1, u_1) R(\theta_2, s) T(-p) 
\]

\subsection{Matrices as Vectors}

If we have vectors $v_1, v_2, v_3$ and a matrix $M$. The $v_1 M$ is a
vector composed of the projection of $v_1$ on each column of $M$. If
we take a matrix formed with $V = (v_1^\top, v_2^\top, v_3^\top)^\top$
then $VM$ has in row $i$ $v_iM$.


\subsection{OpenGL Transforms}

The OpenGL modes are MODELVIEW $M$, PROJECTION $P$, TEXTURE $T$, and
COLOR $C$.

\[
  p' = pMP
\]

I don't known when $T$ and $C$ are applied.

If you apply two transformations $M_1$ followed by $M_2$ then OpenGL
applies them in reverse order, e.g.

\[
  p' = M_2 * M_1
\]

\subsection{Rotation Matrix}

Rotation of an angle $theta$ about a vector $x$ is given by:

\[
  S(x) = \left ( \begin{array}{ccc}
     0    & -x_3 & x_2    \\
     x_3  & 0    & -x_1   \\
     -x_2 & x_1  & 0      \\
  \end{array} \right )
\]

and

\[ u_{i,j} = \left \{ \begin{array}{ll}
     1 & \mbox{if}\:\:\: i = j \\
     0 & \mbox{otherwise}\: 
   \end{array} \right .
\]

then for $i, j$ in $1 .. 3$

\[
  r_{i,j}(\theta, x) =
    x_i x_j + (u_{i,j} - x_ix_j)\cos \theta + s_{i,j}(x) \sin \theta
\]

and otherwise

\[
  r_{i,j} = u_{i4} u_{j4}
\]

This matrix is applied in the following manner.

\[
  p' = R(\theta, x) p^\top
\]

\end{document}
