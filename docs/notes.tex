\documentclass[12pt]{article}

\usepackage{amstex}

\begin{document}

This article is trying to get in place the basic mathematics to
understand how to use OpenGL.

We use row vectors, e.g. $x$, and denote associated column vector
$x\topT$. We work in a homoginous coorindate system rather than the
cartesian space.

\[
   x = (x_1,\dots,x_4)
\]

The projection from homoginous coordinates to cartesian space is

\[
   x \mapsto \frac{1}{x_4}(x_1, x_2, x_3)
\]

Direction vectors have unit length. We can direction vectors for the 4
axis. $i_1,\dots,i_3$ for the three major direction. Matricies
are indicated by upper case.

\[
   M = \left ( \begin{array}{ccc}
     m_{1,1} & \dots & m_{4,1}  \\
     \vdots & \dots & \vdots  \\
     m_{4,1} & \dots & m_{4,4}
     \end{array}
   \right )
\]

The camera has a view point defined by a position, $p$,
forward direction $f$ and upward direction $u$. Moving
forward a scalar distance $d$ is

\[
   p = p + df
\]

The sideways direction $s$ is defined as $f \times u$. A translation
matrix for translation in the $x$ direction is

\[
  T(x) = \left ( 
  \begin{array}{cccc}
     1 & 0 & 0 & x_1      \\
     0 & 1 & 0 & x_2      \\
     0 & 0 & 1 & x_3      \\
     0 & 0 & 0 & 1
  \end{array}
  \right )
\]

Matricies are applied on the left so

\[
  T(x) p^\top = (p_1 + p_4 x_1, p_2 + p_4 x_2, p_3 + p_4 x_3, p_4)
\]

The projection back into 3 space gives

\[
  (p_i / p_4 + p_4 x_i / p_4)_{i} = (p_i/p_4 + x_i)_{i}
\]

The mouse moves in two dimensions $r = (r_1, r_2, 0, 0)$. We translate
these directions to angles $\theta = (\theta_1, \theta_2, \theta_3, 0)$. The
character has a position $p$.

Let $M(\theta, i)$ be a rotation around the axis $i$. The view
transform is then

\[
  T(p) R(\theta_2, s) R(\theta_1, u_1) 
\]

where $s = u_2 R(\theta_1, u_1)$.

This is the transform of the eye from it's initial position. For
OpenGL we need the inverse. Transformations and rotations have an easy
inverse you just negate vector or the angle.

\[
   (AB)^{-1} = B^{-1} A^{-1}
\]

Therefore we obtain the 

\[
  R(-\theta_1, u_1) R(\theta_2, s) T(-p) 
\]

\subsection{Matrices as Vectors}

If we have vectors $v_1, v_2, v_3$ and a matrix $M$. The $v_1 M$ is a
vector composed of the projection of $v_1$ on each column of $M$. If
we take a matrix formed with $V = (v_1^\top, v_2^\top, v_3^\top)^\top$
then $VM$ has in row $i$ $v_iM$.


\subsection{OpenGL Transforms}

The OpenGL modes are MODELVIEW $M$, PROJECTION $P$, TEXTURE $T$, and
COLOR $C$.

\[
  p' = pMP
\]

I don't known when $T$ and $C$ are applied.

If you apply two transformations $M_1$ followed by $M_2$ then OpenGL
applies them in reverse order, e.g.

\[
  p' = M_2 * M_1
\]

\subsection{Rotation Matrix}

Rotation of an angle $theta$ about a vector $x$ is given by:

\[
  S(x) = \left ( \begin{array}{ccc}
     0    & -x_3 & x_2    \\
     x_3  & 0    & -x_1   \\
     -x_2 & x_1  & 0      \\
  \end{array} \right )
\]

and

\[ u_{i,j} = \left \{ \begin{array}{ll}
     1 & \mbox{if}\:\:\: i = j \\
     0 & \mbox{otherwise}\: 
   \end{array} \right .
\]

then for $i, j$ in $1 .. 3$

\[
  r_{i,j}(\theta, x) =
    x_i x_j + (u_{i,j} - x_ix_j)\cos \theta + s_{i,j}(x) \sin \theta
\]

and otherwise

\[
  r_{i,j} = u_{i4} u_{j4}
\]

This matrix is applied in the following manner.

\[
  p' = R(\theta, x) p^\top
\]

\end{document}
